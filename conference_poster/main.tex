%%%%%%%%%%%%%%%%%%%%%%%%%%%%%%%%%%%%%%%%%
% Jacobs Landscape Poster
% LaTeX Template
% Version 1.1 (14/06/14)
%
% Created by:
% Computational Physics and Biophysics Group, Jacobs University
% https://teamwork.jacobs-university.de:8443/confluence/display/CoPandBiG/LaTeX+Poster
% 
% Further modified by:
% Nathaniel Johnston (nathaniel@njohnston.ca)
%
% This template has been downloaded from:
% http://www.LaTeXTemplates.com
%
% License:
% CC BY-NC-SA 3.0 (http://creativecommons.org/licenses/by-nc-sa/3.0/)
%
%%%%%%%%%%%%%%%%%%%%%%%%%%%%%%%%%%%%%%%%%

%----------------------------------------------------------------------------------------
%	PACKAGES AND OTHER DOCUMENT CONFIGURATIONS
%----------------------------------------------------------------------------------------

\documentclass[final]{beamer}

\usepackage[orientation=portrait, size=a0, scale=1.3]{beamerposter} % Use the beamerposter package for laying out the poster
\usepackage{lipsum}
\usepackage{xspace}

\usepackage{ragged2e}

\usepackage{listings}
\usepackage{lstrfun}

\usepackage{xparse}
\makeatletter
\RenewDocumentCommand\beamer@newblock{u{\newblock}}{%
  \def\newblock{%
    #1%
    \usebeamercolor[fg]{bibliography entry author}%
    \usebeamerfont{bibliography entry author}%
    \usebeamertemplate{bibliography entry author}%
    \def\newblock{%
      \usebeamercolor[fg]{bibliography entry title}%
      \usebeamerfont{bibliography entry title}%
      \usebeamertemplate{bibliography entry title}%
      \def\newblock{%
      \usebeamercolor[fg]{bibliography entry location}%
      \usebeamerfont{bibliography entry location}%
      \usebeamertemplate{bibliography entry location}%
      \def\newblock{%
        \usebeamercolor[fg]{bibliography entry note}%
        \usebeamerfont{bibliography entry note}%
        \usebeamertemplate{bibliography entry note}}}}}%
  \leavevmode\setbox\beamer@tempbox=\hbox{}\ht\beamer@tempbox=1.5em\box\beamer@tempbox\newblock}
\makeatother

\lstset{language=RFun}

\usetheme{confposter} % Use the confposter theme supplied with this template

\setbeamercolor{block title}{fg=black,bg=white} % Colors of the block titles
\setbeamercolor{block body}{fg=black,bg=white} % Colors of the body of blocks
\setbeamercolor{block alerted title}{fg=white,bg=KUred} % Colors of the highlighted block titles
\setbeamercolor{block alerted body}{fg=black,bg=KUred!10} % Colors of the body of highlighted blocks
% Many more colors are available for use in beamerthemeconfposter.sty

%-----------------------------------------------------------
% Define the column widths and overall poster size
% To set effective sepwid, onecolwid and twocolwid values, first choose how many columns you want and how much separation you want between columns
% In this template, the separation width chosen is 0.024 of the paper width and a 4-column layout
% onecolwid should therefore be (1-(# of columns+1)*sepwid)/# of columns e.g. (1-(4+1)*0.024)/4 = 0.22
% Set twocolwid to be (2*onecolwid)+sepwid = 0.464
% Set threecolwid to be (3*onecolwid)+2*sepwid = 0.708

\newlength{\sepwid}
\newlength{\onecolwid}
\newlength{\twocolwid}
\newlength{\threecolwid}
\setlength{\hoffset}{-1in} % (1 inch) 2,54 + hoffset = 4
% \addtolength{\oddsidemargin}{0cm}
\setlength{\oddsidemargin}{4cm}

% \setlength{\voffset}{0cm} % (1 inch) 2,54 + hoffset = 4
\addtolength{\textwidth}{-8cm} %
% \setlength{\paperwidth}{841mm} % A0 width: 46.8in
% \setlength{\paperheight}{1189mm} % A0 height: 33.1in
\setlength{\sepwid}{0.05\textwidth} % Separation width (white space) between columns
\setlength{\onecolwid}{.30\textwidth} % Width of one column
\setlength{\twocolwid}{0.65\textwidth} % Width of two columns
%-----------------------------------------------------------

\usepackage{graphicx}  % Required for including images

\usepackage{booktabs} % Top and bottom rules for tables

%----------------------------------------------------------------------------------------
%	TITLE SECTION
%----------------------------------------------------------------------------------------

\title{Design of quantum circuits with RFun} % Poster title

\author{Michael Kirkedal Thomsen} % Author(s)

\institute{m.kirkedal@di.ku.dk} % Institution(s)

%----------------------------------------------------------------------------------------

\begin{document}


\newcommand{\rfun}{RFun\xspace}


\defverbatim[colored]\RevRFunPlus{%
\footnotesize
  \begin{rfuncode}
zip :: ([a], [b]) <-> [(a, b)]
zip ([], []) = []
zip ((a:as), (b:bs)) =
  let ls = zip (as, bs)
  in ((a, b):ls)

unzip :: [(a, b)] <-> ([a], [b])
unzip l = zip! l
  \end{rfuncode}
}


\defverbatim[colored]\RevRFunMap{%
\footnotesize
  \begin{rfuncode}
map :: (a <-> b) -> [a] <-> [b]
map fun [ ] = [ ]
map fun (l:ls) =
  let l'  = fun l
      ls' = map fun ls
  in  (l':ls')
  \end{rfuncode}
}


\defverbatim[colored]\RevRFunQC{%
\footnotesize
  \begin{rfuncode}
-- Wire is defined as a number
data Wire = Z | S Wire

data Gate = Id Wire | Not Wire
   | H Wire | T Wire | Cnot ([Wire], Wire)

hadamard :: Wire <-> Gate
hadamard w = (H w)

cnot :: Wire -> Wire <-> Gate
cnot q1 q2 = Cnot [q1] q2

-- Simple example circuit
qec :: ([Wire] <-> [Gate]) 
      -> [Wire] <-> [Gate]
qec phase c =
  let ch = map hadamard c
      p  = phase ch
      ci = map hadamard c
  in  ci
\end{rfuncode}
}



%%%% IFL LANG

\renewcommand{\o}{\texttt{;}}
\renewcommand{\a}{\alpha}
\renewcommand{\i}[1]{#1^{\mbox{-}1}}
\newcommand{\rupd}{\rightarrow}
\newcommand{\cgen}{\;\texttt{?}\;}
\newcommand{\cspe}{\;\texttt{?}\;}
\newcommand{\crev}{\;\texttt{!}\;}
\newcommand{\fdown}[1]{\diagdown}
\newcommand{\fup}[1]{\diagup}
\newcommand{\id}{\ensuremath{\mathsf{Id}}}
\newcommand{\cnot}{\ensuremath{\mathsf{Cnot}}}
\newcommand{\hgate}{\ensuremath{\mathsf{H}}}
\newcommand{\tgate}{\ensuremath{\mathsf{T}}}
\newcommand{\notg}{\ensuremath{\mathsf{Not}}}
\newcommand{\zip}{\ensuremath{\mathsf{Zip}}}
\newcommand{\concat}{\ensuremath{\mathsf{Concat}}}
\newcommand{\spli}{\ensuremath{\mathsf{Split}}}
\newcommand{\flip}{\ensuremath{\mathsf{Flip}}}
\newcommand{\rup}{\ensuremath{\mathsf{Rup}}}
\newcommand{\rdown}{\ensuremath{\mathsf{Rdn}}}
\newcommand{\lif}{\;\mathrm{if}\;}
\newcommand{\lelse}{\;\mathrm{else}\;}


%%%%%




\usebeamerfont{block body}

\addtobeamertemplate{block end}{}{\vspace*{2ex}} % White space under blocks
\addtobeamertemplate{block alerted end}{}{\vspace*{2ex}} % White space under highlighted (alert) blocks

\setlength{\belowcaptionskip}{2ex} % White space under figures
\setlength\belowdisplayshortskip{2ex} % White space under equations

\begin{frame}[t] % The whole poster is enclosed in one beamer frame
%
\begin{columns}[t,onlytextwidth] % The whole poster consists of three major columns, the second of which is split into two columns twice - the [t] option aligns each column's content to the top

\begin{column}{\onecolwid} % The first column

%----------------------------------------------------------------------------------------
%	OBJECTIVES
%----------------------------------------------------------------------------------------

\begin{alertblock}{Abstract}

Languages for describing quantum circuits~\cite{ChongEtAl:2017:Qlang} follow standard HDLs with extension for quantum constructs, thus not exploiting the properties of the quantum logic model.
%
Here we explore how DSLs for a reversible model can be transferred to design of quantum circuits.

Specifically, we implement a quantum extension to combinator language \cite{Thomsen:2012:IFL} in a reversible functional language, \rfun. This gives the strength of circuits described in a combinator languages with type safety of \rfun.

% I will also expand it with quantum gates. I falls out quite nicely.

% \begin{itemize}
% \item Mollis dignissim, magna augue tincidunt dolor, interdum vestibulum urna
% \item Sed aliquet luctus lectus, eget aliquet leo ullamcorper consequat. Vivamus eros sem, iaculis ut euismod non, sollicitudin vel orci.
% \item Nascetur ridiculus mus.
% \item Euismod non erat. Nam ultricies pellentesque nunc, ultrices volutpat nisl ultrices a.
% \end{itemize}

\end{alertblock}

\end{column}

\begin{column}{\sepwid}~\end{column} % Empty spacer column

\begin{column}{\twocolwid} % Begin a column which is two columns wide (column 2)

%----------------------------------------------------------------------------------------
% MATERIALS
%----------------------------------------------------------------------------------------

\textbf{\large \rfun, a Reversible Functional Language}

\begin{columns}[t,onlytextwidth]

\begin{column}{\onecolwid} % The first column
  \begin{block}{}
  \vspace{-22mm}
  \rfun \cite{YokoyamaAxelsenGlueck:2012:LNCS,KaarsgaardThomsen:2017:NWPT} is a \emph{history-free} reversible functional language. Fundamentally, a reversible computation can be considered as an injective transformation of a state into an updated state, thus \rfun implements (often) injective \emph{partial} functions.

  Noticeable for \rfun is a type system supporting:
\begin{itemize}
  \item linear usage of resources -- no duplication
  \item ancilla usage -- guaranteed restore
  \item local inverse semantics -- invertibility
\end{itemize}

% The materials were prepared according to the steps outlined below:

\textbf{Tutorial and interpreter found at:}
\url{http://bit.ly/rfun-lang}


  \end{block}
\end{column}

%----------------------------------------------------------------------------------------

\begin{column}{\sepwid}~\end{column} % Empty spacer column

\begin{column}{\onecolwid} % The first column

\RevRFunPlus

~

\RevRFunMap


\end{column} % End of column 2.1

\end{columns}

\end{column} % End of column 2.1

\end{columns} % End of the split of column 2 - any content after this will now take up 2 columns width


%----------------------------------------------------------------------------------------
%	CONTENT
%----------------------------------------------------------------------------------------

\vspace{20mm}
\textbf{\large Describing Quantum Circuits in \rfun as Combinators}

\begin{columns}[t,onlytextwidth] % The whole poster consists of three major columns, the second of which is split into two columns twice - the [t] option aligns each column's content to the top

\begin{column}{\onecolwid} % The first column

\begin{block}{}
  \vspace{-22mm}
Here we present and implements a combinator-style functional language designed to be close to the quantum logical gate-level. The combinators include high-level constructs such as ripples, but also the recognisable inversion combinator $\i{f}$, which defines the inverse function of $f$ using an efficient semantics.

It is important to ensure that all circuits descriptions follows model constraints, and furthermore we must require this to be done statically. This is ensured by the type system, which also allows the description of arbitrary sized circuits. The combination of the functional language and the restricted reversible model results in many arithmetic laws, which provide more possibilities for term rewriting and, thus, the opportunity for good optimisation.
\end{block}

\end{column}

\begin{column}{\sepwid}~\end{column} % Empty spacer column

\begin{column}{\onecolwid}

\begin{block}{}
  \vspace{-22mm}
The combinator language is based on the following slightly simplified syntax.

\vspace{-15mm}
\begin{figure}
\begin{align*}
D ::=&\; (\mathit{funcName} = R)^* &~& \mathrm{definition} \\
R ::=&\; \id  \;|\; \notg \;|\; \hgate \;|\; \tgate \;|\; ... &&\mathrm{basic\;gates}  \\
     % &\; \flip\;|\; \rup \;|\; \rdown \;|\; \{n_1,n_2,\dots\} && \mathrm{permutations} \\
     &\; \zip \;|\; \spli \;|\; \concat  && \mathrm{reordering} \\
     &\; \mathit{funcName}  && \mathrm{function\;use} \\
     &\; R \o R \;|\; [R,R] && \mathrm{composition} \\
     &\; \a R \;|\;\fdown{} R \;|\; \fup{} R && \mathrm{map,\;ripples} \\%and\;rippling/folding} \\
     &\; \i{R}  && \mathrm{inverse}
\end{align*}
\caption{Syntax for central subset of combinator language.}
\end{figure}

Note, that this includes only resource preserving combinators and basic quantum gates and, thus, is a restriction to unitarity operations. Measurements and setup should be handled externally.

The following implementation examples define the combinator language in \rfun.

\end{block}

\end{column}

\begin{column}{\sepwid}~\end{column} % Empty spacer column

\begin{column}{\onecolwid}

\RevRFunQC

\end{column}

\end{columns}

\vspace{10mm}
\textbf{\large References}

\bibliographystyle{acm}
\bibliography{references}

















% \begin{columns}[t,onlytextwidth] % The whole poster consists of three major columns, the second of which is split into two columns twice - the [t] option aligns each column's content to the top

% \begin{column}{\onecolwid} % The first column



% \begin{block}{Introduction}
% Lorem ipsum dolor \textbf{sit amet}, consectetur adipiscing elit. Sed commodo molestie porta. Sed ultrices scelerisque sapien ac commodo. Donec ut volutpat elit. Sed laoreet accumsan mattis. Integer sapien tellus, auctor ac blandit eget, sollicitudin vitae lorem. Praesent dictum tempor pulvinar. Suspendisse potenti. Sed tincidunt varius ipsum, et porta nulla suscipit et. Etiam congue bibendum felis, ac dictum augue cursus a. \textbf{Donec} magna eros, iaculis sit amet placerat quis, laoreet id est. In ut orci purus, interdum ornare nibh. Pellentesque pulvinar, nibh ac malesuada accumsan, urna nunc convallis tortor, ac vehicula nulla tellus eget nulla. Nullam lectus tortor, \textit{consequat tempor hendrerit} quis, vestibulum in diam. Maecenas sed diam augue.

% This statement requires citation \cite{Smith:2012qr}.
% \end{block}



% %------------------------------------------------

% \begin{figure}
% \includegraphics[width=0.8\linewidth]{placeholder.jpg}
% \caption{Figure caption}
% \end{figure}

% %----------------------------------------------------------------------------------------

% \end{column} % End of the first column

% \begin{column}{\sepwid}\end{column} % Empty spacer column

% \begin{column}{\twocolwid} % Begin a column which is two columns wide (column 2)

% %----------------------------------------------------------------------------------------
% %	IMPORTANT RESULT
% %----------------------------------------------------------------------------------------

% \begin{alertblock}{Important Result}

% Lorem ipsum dolor \textbf{sit amet}, consectetur adipiscing elit. Sed commodo molestie porta. Sed ultrices scelerisque sapien ac commodo. Donec ut volutpat elit.

% \end{alertblock} 

% %----------------------------------------------------------------------------------------

% \begin{columns}[t,totalwidth=\twocolwid] % Split up the two columns wide column again

% \begin{column}{\onecolwid} % The first column within column 2 (column 2.1)

% %----------------------------------------------------------------------------------------
% %	MATHEMATICAL SECTION
% %----------------------------------------------------------------------------------------

% \begin{block}{Mathematical Section}

% Nam quis odio enim, in molestie libero. Vivamus cursus mi at nulla elementum sollicitudin. Nam quis odio enim, in molestie libero. Vivamus cursus mi at nulla elementum sollicitudin.
  
% \begin{equation}
% E = mc^{2}
% \label{eqn:Einstein}
% \end{equation}

% Nam quis odio enim, in molestie libero. Vivamus cursus mi at nulla elementum sollicitudin. Nam quis odio enim, in molestie libero. Vivamus cursus mi at nulla elementum sollicitudin.

% \begin{equation}
% \cos^3 \theta =\frac{1}{4}\cos\theta+\frac{3}{4}\cos 3\theta
% \label{eq:refname}
% \end{equation}

% Nam quis odio enim, in molestie libero. Vivamus cursus mi at nulla elementum sollicitudin. Nam quis odio enim, in molestie libero. Vivamus cursus mi at nulla elementum sollicitudin.

% \begin{equation}
% \kappa =\frac{\xi}{E_{\mathrm{max}}} %\mathbb{ZNR}
% \end{equation}

% \end{block}

% %----------------------------------------------------------------------------------------

% \end{column} % End of column 2.1

% \begin{column}{\onecolwid} % The second column within column 2 (column 2.2)

% %----------------------------------------------------------------------------------------
% %	RESULTS
% %----------------------------------------------------------------------------------------

% \begin{block}{Results}

% \begin{figure}
% \includegraphics[width=0.8\linewidth]{placeholder.jpg}
% \caption{Figure caption}
% \end{figure}

% Nunc tempus venenatis facilisis. Curabitur suscipit consequat eros non porttitor. Sed a massa dolor, id ornare enim:

% \begin{table}
% \vspace{2ex}
% \begin{tabular}{l l l}
% \toprule
% \textbf{Treatments} & \textbf{Response 1} & \textbf{Response 2}\\
% \midrule
% Treatment 1 & 0.0003262 & 0.562 \\
% Treatment 2 & 0.0015681 & 0.910 \\
% Treatment 3 & 0.0009271 & 0.296 \\
% \bottomrule
% \end{tabular}
% \caption{Table caption}
% \end{table}

% \end{block}

% %----------------------------------------------------------------------------------------

% \end{column} % End of column 2.2

% \end{columns} % End of the split of column 2

% \end{column} % End of the second column

% % \begin{column}{\sepwid}\end{column} % Empty spacer column

% % \begin{column}{\onecolwid} % The third column

% % %----------------------------------------------------------------------------------------
% % %	CONCLUSION
% % %----------------------------------------------------------------------------------------

% % \begin{block}{Conclusion}

% % Nunc tempus venenatis facilisis. \textbf{Curabitur suscipit} consequat eros non porttitor. Sed a massa dolor, id ornare enim. Fusce quis massa dictum tortor \textbf{tincidunt mattis}. Donec quam est, lobortis quis pretium at, laoreet scelerisque lacus. Nam quis odio enim, in molestie libero. Vivamus cursus mi at \textit{nulla elementum sollicitudin}.

% % \end{block}

% % %----------------------------------------------------------------------------------------
% % %	ADDITIONAL INFORMATION
% % %----------------------------------------------------------------------------------------

% % \begin{block}{Additional Information}

% % Maecenas ultricies feugiat velit non mattis. Fusce tempus arcu id ligula varius dictum. 
% % \begin{itemize}
% % \item Curabitur pellentesque dignissim
% % \item Eu facilisis est tempus quis
% % \item Duis porta consequat lorem
% % \end{itemize}

% % \end{block}

% % %----------------------------------------------------------------------------------------
% % %	REFERENCES
% % %----------------------------------------------------------------------------------------

% % \begin{block}{References}

% % \nocite{*} % Insert publications even if they are not cited in the poster
% % \small{\bibliographystyle{unsrt}
% % \bibliography{sample}\vspace{0.75in}}

% % \end{block}

% % %----------------------------------------------------------------------------------------
% % %	ACKNOWLEDGEMENTS
% % %----------------------------------------------------------------------------------------

% % \setbeamercolor{block title}{fg=red,bg=white} % Change the block title color

% % \begin{block}{Acknowledgements}

% % \small{\rmfamily{Nam mollis tristique neque eu luctus. Suspendisse rutrum congue nisi sed convallis. Aenean id neque dolor. Pellentesque habitant morbi tristique senectus et netus et malesuada fames ac turpis egestas.}} \\

% % \end{block}

% % %----------------------------------------------------------------------------------------
% % %	CONTACT INFORMATION
% % %----------------------------------------------------------------------------------------

% % \setbeamercolor{block alerted title}{fg=black,bg=norange} % Change the alert block title colors
% % \setbeamercolor{block alerted body}{fg=black,bg=white} % Change the alert block body colors

% % \begin{alertblock}{Contact Information}

% % \begin{itemize}
% % \item Web: \href{http://www.university.edu/smithlab}{http://www.university.edu/smithlab}
% % \item Email: \href{mailto:john@smith.com}{john@smith.com}
% % \item Phone: +1 (000) 111 1111
% % \end{itemize}

% % \end{alertblock}

% % \begin{center}
% % \begin{tabular}{ccc}
% % \includegraphics[width=0.4\linewidth]{logo.png} & \hfill & \includegraphics[width=0.4\linewidth]{logo.png}
% % \end{tabular}
% % \end{center}

% % %----------------------------------------------------------------------------------------

% % \end{column} % End of the third column

% \end{columns} % End of all the columns in the poster

\end{frame} % End of the enclosing frame

\end{document}
